\section{Introducción}
% Especificar la motivación y las características principales del proyecto como así también un marco general
% sobre el estado del arte de la tecnología y el uso de las ideas o conceptos del proyecto propuesto.
Informar a las personas de hechos que sucedieron o sucederán en un ámbito especifico es una necesidad por parte de la sociedad en la que vivimos. Ya que tomamos decisiones según la información que obtenemos a través de los medios de comunicación.

En las facultades de la UNLP se consume muchos recursos para cumplir este fin como ser afiches, pancartas, panfletos, etc. los cuales no tiene una vida útil prolongada. Actualmente no se tiene en cuenta el impacto ambiental que genera esta forma de comunicación.
Así mismo, el material en cuestión (mayormente papel) se desecha ya que no hay una forma de reutilizarlo, aumentando la cantidad de basura notablemente.

Además, el inconveniente de los afiches en las facultades no solo es ambiental, sino que presenta una polución visual considerable, por la gran cantidad de estos distribuidos en todos los lugares transitables.

Partiendo de este hecho, la forma de comunicación debe ser más limpia acorde a forma de mostrar la información.
Por ello, este proyecto se basa en satisfacer la necesidad sin tener un impacto ambiental negativo.

\section{Objetivos del proyecto}
% Deben especificarse los objetivos que se plantearon en el informe inicial
% (primarios y secundarios) se hayan cumplido o no (en las conclusiones se deberá contemplar
% el cambio de objetivos o el grado de cumplimiento de los mismos).
El objetivo principal del proyecto es implementar un sistema el cual controla un cartel LED de forma remota.
El mismo se puede subdividir en objetivos funcionales los cuales se mencionan a continuación.

Diseñar e implementar el Hardware (PCB, Cartel) modularizable, para que se posible agregar módulos de LED y expandir la cantidad de caracteres en una renglón.

Diseñar e implementar protocolo de comunicación..

Desarrollar el software que se ejecutará en el microcontrolador..

Desarrollar el software que controla el cartel, éste podrá ser usado dede una pc y tendrá características de un panel para controlar la funcionalidad.

\section{Análisis de requerimientos}
% Enumerar los requerimientos funcionales y  no funcionales sobre los que se diseñó la solución propuesta. 
% Especificar los aspectos funcionales que el sistema deberá cumplir respecto del usuario y su interacción con el mismo.

% Hablar de limites de tiempos (tiempo máximo de respuesta de conexión)

\section{Diseño del hardware}
% Describir el hardware propuesto, comenzando por un diagrama en bloques, luego una descripción de las Interfaces eléctricas
% (conexiones poncho-EDU-CIAA) y de usuario (teclado, displays, etc)
% hasta llegar al detalle fino de los componentes y circuitos a utilizar y sus características.
El cartel se componen de módulos, cada módulo contiene una matriz de LED 8x8, un chip shifteador \cite{MAX}, entre otros componentes, logrando así una unidad funcional capas de mostrar un carácter configurado por software.


\section{Diseño de software}
% Descripción del software propuesto, comenzando por un diagrama de estados, pseudocódigo o esquemas que permitan
% comprender la arquitectura de software elegida (Controlada por eventos o por tiempo, apropiativa, cooperativa, etc).
% Luego especificar las diferentes módulos (o tareas) a implementar, drivers, bibliotecas a utilizar,
% prioridades y planificación de las distintas tareas (No incluir en el informe código C).

\section{Ensayos y mediciones}
% Describir el proceso de construcción del prototipo y los ensayos realizados para verificar el funcionamiento del mismo.
% Mostrar mediciones o resultados obtenidos en forma de tablas, figuras, 
% fotografías del sistema, pantallas de osciloscopios, pantallas de display, etc

\section{Conclusiones parciales}
% 1-Explicar el grado de avance de los objetivos planteados hasta el momento como así también el trabajo a seguir.
% 2-Describir claramente la actividad de cada integrante del grupo, evaluar las horas invertidas hasta el momento
%	y re-estimar las horas de ingeniería restantes.
% 3- Evaluar y destacar correcciones o desvíos del cronograma de tareas presentado en el informe de inicial.
% 4-Analizar el presupuesto invertido y corregir (si corresponde) el presupuesto final del proyecto. 