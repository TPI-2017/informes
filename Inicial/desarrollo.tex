\section{Introducción}
% Especificar la motivación y las características principales del proyecto como así también un marco general
% sobre el estado del arte de la tecnología y el uso de las ideas o conceptos del proyecto propuesto.

\section{Objetivos del proyecto}
% Deben especificarse los objetivos que se plantearon en el informe inicial
% (primarios y secundarios) se hayan cumplido o no (en las conclusiones se deberá contemplar
% el cambio de objetivos o el grado de cumplimiento de los mismos).

\section{Análisis de requerimientos}
% Enumerar los requerimientos funcionales y  no funcionales sobre los que se diseñó la solución propuesta. 
% Especificar los aspectos funcionales que el sistema deberá cumplir respecto del usuario y su interacción con el mismo.

\section{Diseño del hardware}
% Describir el hardware propuesto, comenzando por un diagrama en bloques, luego una descripción de las Interfaces eléctricas
% (conexiones poncho-EDU-CIAA) y de usuario (teclado, displays, etc)
% hasta llegar al detalle fino de los componentes y circuitos a utilizar y sus características.

\section{Diseño de software}
% Descripción del software propuesto, comenzando por un diagrama de estados, pseudocódigo o esquemas que permitan
% comprender la arquitectura de software elegida (Controlada por eventos o por tiempo, apropiativa, cooperativa, etc).
% Luego especificar las diferentes módulos (o tareas) a implementar, drivers, bibliotecas a utilizar,
% prioridades y planificación de las distintas tareas (No incluir en el informe código C).

\section{Ensayos y mediciones}
% Describir el proceso de construcción del prototipo y los ensayos realizados para verificar el funcionamiento del mismo.
% Mostrar mediciones o resultados obtenidos en forma de tablas, figuras, fotografías del sistema, pantallas de osciloscopios, pantallas de display, etc

\section{Conclusiones parciales}
% 1-Explicar el grado de avance de los objetivos planteados hasta el momento como así también el trabajo a seguir.
% 2-Describir claramente la actividad de cada integrante del grupo, evaluar las horas invertidas hasta el momento
%	y re-estimar las horas de ingeniería restantes.
% 3- Evaluar y destacar correcciones o desvíos del cronograma de tareas presentado en el informe de inicial.
% 4-Analizar el presupuesto invertido y corregir (si corresponde) el presupuesto final del proyecto. 